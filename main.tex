\documentclass[12pt,dvipdfmx]{report}
\usepackage[hmargin=1.8cm,vmargin=1.4cm,includehead,includefoot]{geometry}
\usepackage{graphicx}
\usepackage{pdfpages}
\usepackage{fancyhdr}
\usepackage{hyperref}

% 寄稿記事の挿入のためのマクロ
% \article{タイトル}{著者}{ファイル名}
\newcommand{\article}[3]{
    \fancyfoot[L]{#2} % Author name on the right footer
    \phantomsection % hyperrefのためのダミー
    \addcontentsline{toc}{section}{#1\\ #2}% 個々のタイトルなどは目次ではsection扱い
    \includepdf[pages=-, pagecommand={\thispagestyle{fancy}}]{#3} % 原稿のPDFファイルを挿入
}

\begin{document}

%
% 表紙
%
\setcounter{page}{0}
\thispagestyle{empty}

\vspace*{1.5cm}

\begin{center}
\Huge{\sffamily Institute for Mathematical Informatics}\\
\vspace{1.5cm}
\Huge{\sffamily Annual Report}\\
\vspace{0.5cm}
\Huge{\sffamily 2024}\\
\vspace{0.5cm}
\Huge{\sffamily Vol. 1}\\

\vfill
\includegraphics[scale=0.75]{circle_logo.pdf}
\vfill

\Huge{M}\Large{EIJI}
\Huge{G}\Large{AKUIN}
\Huge{U}\Large{NIVERSITY}
\end{center}

\vspace{1cm}

\newpage

%
% 目次
%
\tableofcontents

\pagestyle{fancy}
\fancyhead{}
\fancyhead[L]{\usefont{OT1}{phv}{m}{n}MG-IMI Annual Report}
\fancyhead[R]{\usefont{OT1}{phv}{m}{n}Vol.~1 (2024)}
\fancyfoot{}
\fancyfoot[R]{\thepage}
\renewcommand{\footrulewidth}{1pt}
\renewcommand{\headrulewidth}{1pt}

%
% 以下、各記事の挿入(基本的には\articleマクロを使う)
% 各記事のファイル名は、このファイル(main.tex)からの相対パスで指定する
% 例: \article{タイトル}{著者}{eng1/eng1.pdf}
% 原則、筆頭著者名のアルファベット順に並べる
%

%
% Hideki Yukawa's contribution
%
\article{On a Possible Interpretation of the Penetrating Component of the Cosmic Ray}{Hideki Yuakwa}{eng1/eng1.pdf}

%
% Paul Dirac's contribution
%
\article{The Quantum Theory of the Electron}{Paul Dirac}{eng2/eng2.pdf}

%
% Japanese LaTeX
%
\article{人工知能の発展に関する歴史}{GitHub Copilot}{jp/jp.pdf}

%
% Japanese Word
%
\article{人工知能の発展に関する歴史(Word)}{GitHub Copilot}{word/word.pdf}

\end{document}